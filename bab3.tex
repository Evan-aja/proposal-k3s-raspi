\section{Metode Penelitian}

\subsection{Studi Literatur}
Untuk menunjang pengerjaan penelitian ini, perlu dilakukan studi literatur untuk memberikan pengetahuan yang cukup untuk menyelesaikan penelitian ini. Studi literatur dilakukan dengan cara mengumpulkan dokumentasi dan pustaka yang berkaitan dengan penelitian ini. yaitu:
\begin{enumerate}
    \item Ansible
    \item K3s
    \item Raspberry Pi
\end{enumerate}

\subsection{Tipe Penelitian}
Tipe dari penelitian ini adalah penelitian eksploratif dan implementatif. Dimana akan dilakukan eksplorasi mengenai langkah-langkah yang dapat diambil untuk mengimplementasikan \textit{tech stack} hingga menjadi kluster yang dapat digunakan dalam lingkungan produksi.

\subsection{Rekayasa Kebutuhan}
\begin{table}[h]
    \begin{tblr}{
            hlines,
            vlines,
            row{1} = {bg=gray,fg=white,ht=1cm},
            columns = {halign=c},
            colspec = {Q[5cm]},
        }
        \textbf{Proses/ Kebutuhan} & \textbf{Perangkat Lunak/Keras}\\
        Node Master/Slave & Raspberry Pi \\
        Orchestrator & K3s \\
        Automation & Ansible \\
    \end{tblr}
    \centering
\caption{Tabel kebutuhan penelitian}
\end{table}

\subsection{Rancangan Sistem}
Perancangan sistem dilakukan sebagai tahap awal untuk menggambarkan implementasi sisten secara sistematis dan terstruktur. Hal ini akan dilakukan ketika seluruh rekayasa kebutuhan untuk penelitian telah terpenuhi.
\begin{figure}[htb!]
    \centering
    \begin{tabular}{ @{} r @{} }
        \includegraphics[scale=0.7]{pictures/rancangan_sistem.png}\\
    \end{tabular}
    \caption{Rancangan Sistem}
\end{figure}
\FloatBarrier
\noindent
Dari diagram perancangan diatas, terlihat beberapa node yang berupa Raspberry Pi terhubung pada sebuah router. Dimana router tersebut menghubungkan semua node kepada sebuah komputer manajemen untuk administrasi. Setiap node dapat berkomunikasi dengan node lain melalui Master Node menggunakan K3s dan komputer yang melakukan manajemen dapat mengakses secara langsung setiap node melalui jaringan yang disediakan melalui router.
\subsection{Skenario Pengujian dan Analisis Hasil}
% Pengujian kinerja dan analisis hasil 

\subsection{Pembentukan Kesimpulan}