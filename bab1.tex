\section{Pendahuluan}
\subsection{Latar Belakang}
Klustering seiring berjalannya waktu semakin banyak digunakan di kalangan perusahaan tekno-logi. hal tersebut terjadi karena kemudahan yang diberikan oleh teknologi ini, untuk melakukan skalasi layanan dengan terhadap peluncuran perangkat lunak kepada konsumen setiap kali kebutuhan bertambah. Dimana dengan berkembangnya zaman digital, kebutuhan manusia untuk mendapatkan semakin banyak kekuatan komputasi semakin tidak dapat dielakkan. Tetapi terdapat batasan skalasi secara vertikal (menambah RAM, menambah CPU, menambah penyimpanan persisten di dalam satu sistem).

\vspace{0.2cm}
\noindent Dengan harga komoditas yang dibutuhkan sebagai bahan dasar pembuatan produk elektronik, inflasi, dan peningkatan biaya produksi barang elektronik, maka skalasi secara vertikal akan menjadi semakin sulit seiring dengan peningkatan harga komponen yang disebabkan oleh kualitas yang dibutuhkan \cite{electronic_cost}. Hal ini menyebabkan bertambahnya penggunaan perangkat komputasi murah yang dapat dengan mudah di produksi masal. dan diprediksi akan semakin meningkat seiring dengan berjalannya waktu \cite{raspi_market}.

\vspace{0.2cm}
\noindent Dengan meningkatnya penggunaan klustering, diperlukan pula sebuah sistem yang mampu memberikan solusi untuk pengelolaan node pada sebuah kluster yang dapat di otomatisasi. sehingga dapat dilakukan peluncuran aplikasi kepada konsumen secara efektif. Dan sistem tersebut harus mampu mengatasi \textit{error} ketika melakukan tugas yang telah diberikan sebelumnya, sehingga kemungkinan kerusakan terhadap seluruh jajaran sistem dapat dihindari.


\subsection{Rumusan Masalah}

\subsection{Tujuan}

\subsection{Manfaat}

\subsection{Batasan Masalah}